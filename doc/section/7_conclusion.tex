\documentclass[../main.tex]{subfiles}

\begin{document}

\section{Conclusion}
After exploring the use of CNN and logistic regression for image classification, it is safe to say that CNN with an outstanding 98\% accuracy on the facemask outperforms logistic regression which managed to achieve 95\%. Comparing our results with other scientific articles that have explored facemask image classification (examples include \cite{scientific_article_1} and \cite{scientific_article_2}), we find our CNN has close to similar performance. The examples mentioned report accuracy scores of $99.49\%$ and $\sim99\%$ respectively. An accuracy of this magnitude confidently supports the method's relevancy at solving the problem presented in this report.

\subsection{Prospects for the future}
The main prospect for the future would be to dive into the fine tunings of CNNs and explore further how different parameters affect the accuracy, in order to push it closer towards the 100\% goal. The logistic regression was implemented by using \verb|scikit-learn|, which gives less elbow room for optimization, but a good prospect for the future would be to see how complexity in the images affect the result. More research can also be done in trying these methods on larger and more complicated and datasets, and explore how they fare in other classification scenarios.

\end{document}